\documentclass[12pt,a4paper,oneside]{book}
\usepackage[utf8]{inputenc}
\usepackage{amsmath}
\usepackage{amsfonts}
\usepackage{amssymb}
\usepackage{listingsutf8}
\usepackage[utf8]{inputenc}
%%ORT
%metadata para la ORT
\usepackage[unicode,
            pdftex]{hyperref}
\hypersetup{
%acá sólo hay que poner lo que pusimos en la caratula y otras partes 
 pdfauthor={¡¡¡Nombre - 123456!!!},
 pdftitle={¡¡¡Titulo que esta en la caratula!!!},
 pdfsubject={¡¡¡tema/descripción!!!},
 pdfkeywords={¡¡¡palabras clave que ya pusimos!!!}
 }

%%posición de la numeración
\usepackage{fancyhdr}
\fancypagestyle{plain}{% Redefining plain page style
  \fancyhf{} %clear all header and footer fields
  \fancyfoot[RO]{\thepage}
}%
\fancyhf{} %clear all header and footer fields
\fancyfoot[RO]{\thepage}
\renewcommand{\headrulewidth}{0pt}
\renewcommand{\footrulewidth}{0pt}
\pagestyle{fancy}

%%Nombre del índice tabla de contenido
\renewcommand\contentsname{Índice}

%%para incluir imagenes
\usepackage{graphicx}

%%para la bibliografía
%%cambia el nombre
\renewcommand\bibname{Referencias Bibliográficas}
%%la agrega al indice
\usepackage[nottoc,numbib]{tocbibind}

%%para los capítulos
\makeatletter
\def\@makechapterhead#1{%
  \vspace*{50\p@}%
  {\parindent \z@ \raggedright \normalfont
    \ifnum \c@secnumdepth >\m@ne
        \huge\bfseries \space \thechapter\space
    %    \par\nobreak
    %    \vskip 20\p@
    \fi
    \interlinepenalty\@M
    \Huge \bfseries #1\par\nobreak
    \vskip 40\p@
  }}
  \makeatother
  
%%Para hacer Haskell bonito
%%Cambialo a gusto
\lstset{
  frame=none,
  xleftmargin=2pt,
  stepnumber=1,
  numbers=left,
  numbersep=5pt,
  %%numberstyle=\ttfamily\tiny\color[gray]{0.3},
  belowcaptionskip=\bigskipamount,
  captionpos=b,
  escapeinside={*'}{'*},
  language=Haskell,
  tabsize=2,
  emphstyle={\bf},
  commentstyle=\it,
  stringstyle=\mdseries\rmfamily,
  showspaces=false,
  keywordstyle=\small\ttfamily,
  columns=flexible,
  basicstyle=\small\ttfamily,
  showstringspaces=false,
  morecomment=[l]\%,
  %%mio:
  breaklines=true,
}

%%para poner letras no ascii en el código fuente 
\lstset{
  literate=
           {á}{{\'a}}1
           {é}{{\'e}}1
           {í}{{\'i}}1
           {ó}{{\'o}}1
           {ú}{{\'u}}1
           {ñ}{{\~n}}1
           {Á}{{\'A}}1
           {É}{{\'E}}1
           {Í}{{\'I}}1
           {Ó}{{\'O}}1
           {Ú}{{\'U}}1
           {Ñ}{{\~N}}1
}
 

\makeindex
\begin{document}
%%\maketitle


%%portada:
\vspace*{\fill}

\begin{center}

\begin{Large}
\textbf{Universidad ORT Uruguay}

\textbf{Facultad de Ingeniería}
\bigskip\bigskip\bigskip\bigskip\bigskip\bigskip
\end{Large}

\begin{huge}

\textbf{Predicción de la edad en las Redes Sociales}
\end{huge}    
\bigskip\bigskip\bigskip\bigskip\bigskip\bigskip


\begin{Large}
\textbf{Entregado como requisito para la obtención del título de
Master en Ingeniería}
\end{Large}
\bigskip\bigskip\bigskip\bigskip\bigskip\bigskip\bigskip\bigskip

\begin{Large}
\textbf{Verónica Tortorella - 153303}
%el -1 sería el número de alumno de ORT
\bigskip

\textbf{Tutor: Sergio Yovine}
\bigskip\bigskip\bigskip\bigskip\bigskip\bigskip
\end{Large}

\begin{huge}
\textbf{2018}
\end{huge}

\end{center}
\vspace*{\fill}

\thispagestyle{empty}
\newpage
\vspace*{\fill}
Yo, Verónica Tortorella, declaro que el trabajo que se presenta en esta obra es de mi
propia mano. Puedo asegurar que:

- La obra fue producida en su totalidad mientras realizaba el Proyecto;

- Cuando he consultado el trabajo publicado por otros, lo he atribuido con claridad;

- Cuando he citado obras de otros, he indicado las fuentes. Con excepción de estas citas, la obra es enteramente nuestra;

- En la obra, he acusado recibo de las ayudas recibidas;

- Cuando la obra se basa en trabajo realizado conjuntamente con otros, he explicado claramente qué fue contribuido por otros, y qué fue contribuido por mi;

- Ninguna parte de este trabajo ha sido publicada previamente a su entrega, excepto donde se han realizado las aclaraciones correspondientes.
\bigskip\bigskip

\vspace*{\fill}


\newpage 

{\huge\bfseries \space Abstract}
\bigskip
\bigskip

¡¡¡Esto es una plantilla ({\it template}) que sigue el documento 302 de la ORT. La idea es que trabajes sobre esto. Todo lo que tiene tres signos de exclamción de abrir y cerrar debe ser modificado. (Y esos signos deben ser borrados...)!!!

\newpage

{\huge\bfseries \space Palabras clave}
\bigskip
\bigskip

¡¡¡Esta es una palabra clave; esta es otra; hay que poner bastantes; separadas por punto y coma; así lo encuentran!!!

\newpage

\tableofcontents

\chapter{Y ahora empezamos}

Hacemos algunas cosas a modo de ejemplo. Todo esto puede ser borrado EXCEPTO por las líneas 
que dicen: bibliographystyle y bibliography. Esas son necesarias para la bibliografía. Hablando de bibliografía, podes modificar el archivo biblio.bib con nueva bibliografía y después usar las citas así:


Al final de esta frase hay una cita.\cite{citalibro}
En la mitad\cite{citaarticulo} de esta hay otra.


\section{Metemos una sección}

Recorda que hay herramientas como {\tt aspell} para corregir la ortografía de documentos latex.
Si te da problema 


\subsection{Yupi! Una subsección.}

Mirá matemáticas!
\\

$\sqrt{(\lambda x . x x)(\lambda y . y)} = \pi^{\lambda y . y}$

\subsection{Otra subsección}

Mirá código!!

\begin{lstlisting}
data Proof a = Line a | Link a (Comment) (Proof a)
-- Este es un comentario del código ÁÉÍÓÚÑ
\end{lstlisting}\bigskip

\subsection{Y otra más!}

\subsection{Última con lista}

\begin{enumerate}
\item Acá hay un uno
\item Ácá hay un dos
\item Acá hay un tres
\end{enumerate}


\chapter{Hay que meter capítulos}

\chapter{Así se lee más fácil}

\chapter{Palabras finales}

Nunca sobran.

%%bibliography
%% No borrar estas dos líneas. Revisar de tener los archivos IEEETran y biblio.bib en la misma carpeta
%% si no están y no anda, correr el compilador de latex varias veces. Primero el bibtex, luego pdflatex, luego pdflatex otra vez.
\bibliographystyle{IEEEtran}
\bibliography{biblio}

\chapter{Anexos}
\section{Un anexo por acá}

Blablabla

\section{Otro por allá}

Blablablal

\end{document}
